\documentclass[11pt]{exam}
\usepackage[activeacute,spanish]{babel} % Permite el idioma espa\~nol.
\usepackage[latin1]{inputenc}
\usepackage{amsmath,amsfonts}
\usepackage[colorlinks]{hyperref}
\usepackage{graphicx}
\usepackage{hyperref}

%added because of problem between framed and exam package
\newcommand*{\renameenviron}[1]{%
  \expandafter\let\csname exam-#1\expandafter\endcsname
      \csname #1\endcsname
  \expandafter\let\csname endexam-#1\expandafter\endcsname
      \csname end#1\endcsname
  \expandafter\let\csname #1\endcsname\relax
  \expandafter\let\csname end#1\endcsname\relax
}
\renameenviron{framed}
\renameenviron{shaded}
\renameenviron{leftbar}
\usepackage{framed}
\usepackage{hyperref}


\pagestyle{headandfoot}
\spanishdecimal{.}

\begin{document}
\firstpageheadrule
%\firstpagefootrule
%\firstpagefooter{}{Pagina \thepage\ de \pages}{}
\runningheadrule
%\runningfootrule
\lhead{\bf\normalsize Taller Python 2018}
\rhead{\bf\normalsize Tarea \'area Qu�mica/Bioqu�mica}
\cfoot{ }
\lfoot{\tiny EVM}
\begin{flushleft}
\vspace{0.2in}
%\hbox to \textwidth{Nombre: \enspace \hrulefill}
%Nombre : \\
\vspace{0.25cm}
\end{flushleft}
%%%%%%%%%%%%%%%%%%%%%%%%%%%%%%%%%%%%%%%%%%

\begin{center}
\textbf{Fecha M'axima de entrega: Viernes 12 de Enero}
\end{center}
\textbf{Instrucciones:} Resuelva el problema propuesto usando Python. Env'ie todos los archivos necesarios para reproducir sus resultados (archivos de datos, c'odigos .py, notebooks .ipynb, etc.) por email a \texttt{evohringer@udec.cl}.

\bigskip

En esta tarea usted deber� calcular el centro de masa de la mol�cula de caffeina. 


Las coordenadas de  los �tomos que forman esta mol�cula la podr� bajar en \url{https://pubchem.ncbi.nlm.nih.gov/compound/caffeine#section=3D-Conformer} apretando el bot�n \textit{Download} y seleccionando uno de los formatos ofrecidos (recomiendo el formato \textit{sdf}). 


\begin{parts}
\part Obtenga del formato que mas le acomode las coordenadas y el tipo de �tomo (elemento) que componen la mol�cula.

\part Utilizando numpy calcule los componentes $x$, $y$ y $z$ del centro de masa de la mol�cula ($cms_{x,y,z}$) mediante la siguiente f�rmula:
\begin{eqnarray}
cms_{x} = \frac{1}{M}\sum_{i=1}^{N} m_{i} * x_{i}\\
cms_{y} = \frac{1}{M}\sum_{i=1}^{N} m_{i} * y_{i}\\
cms_{z} = \frac{1}{M}\sum_{i=1}^{N} m_{i} * z_{i},
\end{eqnarray}
en donde $m_{i}$ es la masa en g del elemento que representa el �tomo, $x_{i}$ la coordenada $x$ del �tomo $i$,  $M$ la masa total de la mol�cula en g, y $N$ es el n�mero total de �tomos en la mol�cula.
\end{parts}

\end{document} 
