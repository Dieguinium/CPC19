\documentclass[5pt]{exam}
%\usepackage[activeacute,spanish]{babel} % Permite el idioma espa\~nol.
\usepackage[utf8x]{inputenc}
%\usepackage[latin1]{inputenc}
\usepackage{amsmath,amsfonts}
\usepackage[colorlinks]{hyperref}
\usepackage{graphicx}
\usepackage{minted} 

\pagestyle{headandfoot}

\begin{document}

\firstpageheadrule
%\firstpagefootrule
%\firstpagefooter{}{Pagina \thepage\ de \pages}{}
\runningheadrule
%\runningfootrule
\lhead{\bf\normalsize Taller Python 2018}
\rhead{\bf\normalsize Gu\'ia 05}
\cfoot{ }
\lfoot{\tiny GR}
\begin{flushleft}
\vspace{0.2in}
%\hbox to \textwidth{Nombre: \enspace \hrulefill}
%Nombre : \\
\vspace{0.25cm}
\end{flushleft}
%%%%%%%%%%%%%%%%%%%%%%%%%%%%%%%%%%%%%%%%%%

\center{\LARGE \textbf{Guia Scipy}
\begin{questions}

\item  En esta guia usted ejercitará algunas características del modulo scipy. El modulo scipy se importa como:
\begin{minted}{python}
import scipy as sci
\end{minted}
\item Integre la función de Bessels  de primer especie de orden 5 entre 0 y su 3 raiz.

\item Calcule las siguientes integrales numricamente: 
    \begin{itemize}
        \item $ \int_0^{10} x sen(2x^3 -10)dx $
        \item $ \int_{-2}^{28} xe^{-x^2+5}dx $
        \item $ \int_0^{100} 10x^3+x^7 dx  $
    \end{itemize}
    
\item El trabajo de expansión un gas de van der Waals se obtiene calculando la integral:
    \begin{equation}
        w = -\int^i_f PdV
    \end{equation}
La presión se puede calcular mediante la expresión:
\begin{equation}
    P = \frac{nRT}{V- nb} - \frac{an^2}{V^2}
\end{equation}
Dado los siguientes datos:
\begin{itemize}
    \item $R = 8.314JK^{-1} mol^{-1}$
    \item $n = 1 mol$
    \item $a = 0.657Jm^{3} mol^{−2} , b = 5.62 × 10 −5 m^3 mol^{−1}$
    \item $T = 298K$
\end{itemize}
Calcule el trabajo si el gas se expande de 0.1 m$^3$ a 0.6m$^{3}$ manteniendo la Temperatura constante.


\item Dada las siguientes matrices:
\[
\begin{bmatrix}
      10  & 8  & 0 &  1  \\
      1   &  3 & 8 &  9 \\
      5   &  9 & 0 &  1 \\
      6   &  7 & 13 & 9 \\
\end{bmatrix}
B =
\begin{bmatrix}
      9  & 8  & 7 &  4  \\
      3   &  6 & 8 &  7 \\
      5   &  2 & 2 &  1 \\
      6   &  12 & 0 & 9 \\
\end{bmatrix}
\]
Calcule las siguientes operaciones:
\begin{itemize}
    \item Inversa de A
    \item   e$^A$
    \item det(B)
    \item Los valores y vectores propios de A
    \item Escriba un programa que calcula la traza de una matriz.
\end{itemize}

\item La operación de multiplicación de una matriz \textbf{A} de dimensiones $i\times j$ por una  matriz \textbf{B} de 
dimnsiones $j\times k $, donde $j=1\dots m $  está dada por :
\begin{equation}
    \label{} AB = \sum_{j=1}^5 A_{ij}B_{ji}
\end{equation}
Implemente una rutina que hace un calculo de multiplicación de matrices y verifíque su resultado con la función
dot de numpy.

\item Una molcula de agua tiene las siguientes coordenadas cartesianas en \AA ngstrom: 
    \begin{table}[h!]
        \centering
        \begin{tabular}{c|ccc}
            O & 0.0000 & 0.0000 & 0.0000 \\
            H & 0.7586 & 0.0000 & 0.5043 \\
            H & 0.7586 & 0.0000 & -0.5043
        \end{tabular}
    \end{table}
    \begin{enumerate}
        \item Guarde las coordenadas x,y,z y el simbolo atómico en un array de numpy.
        \item Escriba un diccionario que tiene como llaves símbolos atómicos y como valor la masa
              atómica. (O: 15.999 g/mol, H: 1.000 g/mol)
        \item El momento de inercia de una mol\item cula esta dado por:
            \begin{equation}
                I_{xx} = \sum_i^N m_i(y_i^2 + z_i^2) \quad I_{yy} = \sum_i^N m_i(x_i^2 + z_i^2) \quad I_{zz} = \sum_i^N m_i(x_i^2 + y_i^2)
            \end{equation}
        En el caso de los elementos diagonales, y para los fuera de la diagonal:
            \begin{equation}
                I_{xy} = \sum_i^N m_ix_iy_i) \quad I_{xz} = \sum_i^N m_ix_iz_i \quad I_{yz}=\sum_i^N m_iy_iz_i
            \end{equation}
        Donde N es el número de átomos. Calcule el tensor de inercia y encuentre los valor de los
        momentos de inercia principales, los que corresponden a los valores propios del tensor de
        inercia
    \end{enumerate}


\item Resuelva numricamente la siguiente ecuación diferencial:
    \begin{equation}
        \frac{dy}{dx} = x -y \quad y(0) = 1 
    \end{equation}

La solución analítica de la ecuación diferencial es:
\begin{equation}
    y(x) = x -1 +2e^{-x}
\end{equation}[
Grafique tanto la solución num\'erica como la solución analítica entre 0 y 5 y compare ambos
resultados.

\item Considere la siguiente ecuación diferencial de segundo orden para y(x).

\begin{equation}
    y^{''} + 2 y' + 2y = cos(x) \qquad y(0) =0 \quad y'(0) = 0
\end{equation}

Esta ecuación se puede transformar en un sistema de dos ecuaciones de primer orden definiendo
una nueva variable dependiente. Por ejemplo:

\begin{align}
    z' + 2z + 2y &= cos(2x) \\
    z &= y'
\end{align}
Con las condiciones iniciales z(0) = y(0) = 0. Resuelva la ecuación diferencial entre 0 y 10 y
grafique el resultado.


\item  De las siguientes funciones, encuentre los máximos, mínimos y raíces. Grafique cada una para verificar visualmente su resultado.
    \begin{itemize}
        \item $f(x) = 2x^4 + x^3 + −3x^2$
        \item $f(x) = −x^6 − 3x^3 + x + 1$
        \item $f(x) = 4x^5 − 8x^4 − 5x^3 + 10x^2 + x − 2$
    \end{itemize}

\item En el archivo peliculas.txt se encuentran los datos de el largo de las películas en minutos, con 
rating IMDB mayor a 7.0 hechas en el año 2015.
    \begin{parts}
        \part Calcule el promedio del largo de las películas.
        \part Grafique un histograma para las película, y dibuje una curva de distribución normal en el
        mismo gráfico.
        \part Calcule la desviación estándar y la varianza de la distribución.
    \end{parts}

\end{questions}
\end{document} 
