\documentclass[11pt]{exam}
\usepackage[activeacute,spanish]{babel} % Permite el idioma espa\~nol.
\usepackage[utf8]{inputenc}
\usepackage{amsmath,epsfig}
\usepackage[colorlinks]{hyperref}

\usepackage{minted} 
\usemintedstyle{emacs}
\usepackage{tcolorbox} % colores para el fondo
\definecolor{bg}{rgb}{0.95,0.95,0.95} %color de fondo

\pagestyle{headandfoot}
\spanishdecimal{.}

\begin{document}

\firstpageheadrule
\runningheadrule
\lhead{\bf\normalsize Taller Python 2019}
\rhead{\bf\normalsize Gu\'ia 01}
\cfoot{ }
\lfoot{\tiny GR}
\begin{flushleft}
\vspace{0.2in}

\vspace{0.25cm}
\end{flushleft}
%%%%%%%%%%%%%%%%%%%%%%%%%%%%%%%%%%%%%%%%%%

\begin{questions}

\item Abra una consola y ejecute el int'erprete de Python (es decir, ejecute el comando \texttt{python}). 
\begin{parts}
\item ?`Qu'e versi'on de Python est'a instalada?
\item Ejecute los siguientes comandos en forma consecutiva:

\begin{minted}[bgcolor=bg]{python}
x = 1
y = 2
print(x,y)
print("El valor de x es ",x," y el valor de y es ",y)
\end{minted}
\item Ahora ejecute:

\begin{minted}[bgcolor=bg]{python}
sx = str(x)
type(sx)
\end{minted}

?`Qu'e tipo de variable es \texttt{sx}? Entonces, ?`qu'e hace la funci'on \texttt{str()}?
\item Ahora ejecute:

\begin{minted}[bgcolor=bg]{python}
mensaje = "El valor de x es " + str(x) + " y el valor de y es " + str(y)
print(mensaje)
\end{minted}

?`Qu'e diferencia observa en el resultado?
\item Conozca la funci'on \texttt{len()}, para esto ejecute:

\begin{minted}[bgcolor=bg]{python}
n = len(mensaje)
print(n)
type(n)
\end{minted}

?`Qu'e valor entrega la funci'on \texttt{len()} aplicada a un string\footnote{``String"\, es el nombre usado com'unmente para una \textit{cadena} de caracteres alfanum'ericos.}?, ?`Qu'e tipo de variable suministra? (pruebe aplic'andola a otros strings).

\end{parts}
\item Existen diversas operaciones definidas entre distintos tipos de variables. Para aprender c'omo funcionan algunas de ellas defina primero las siguientes variables y verifique su tipo:
\begin{itemize}

\begin{minted}[bgcolor=bg]{python}
a = 3.14
b = 2
c = 5
d = 6+2j
e = "hola "
f = "gente"
g = True
\end{minted}
A continuaci'on imprima el valor y el tipo del resultado de las siguientes operaciones: \texttt{a+b, a+d, a+e, b+c, b+d, b+e, f+e, e+f, a*b, a*d, a*e, b*c, b*d, c*e, e*f, a**b, a**d, a**e, b**c, e**a, e**b, e**f, a/b, a/d, a/e, b/c, b/d, b/e, c/b, d/a, d/b, e/a, e/b, e/f,  a*g, b*g, not(g), g and False, g and True, g or False, g or True}. ?`Cu'ales de estas operaciones no est'an definidas?
\item ?`Qu'e pas'o en los casos \texttt{b/c} y \texttt{c/b}?. Busque en las referencias sugeridas la explicaci'on de este comportamiento.
\item Tambi'en existen operaciones que transforman el tipo de variable. Por ejemplo, como continuaci'on del ejercicio anterior, calcule y verifique el tipo de las siguientes operaciones: \texttt{int(a), float(b), d.real, d.imag, a==b, a>b}.

\item Cree un programa Python llamado \texttt{test01.py} e incluya como primeras l'ineas el siguiente c'odigo:

\begin{minted}[bgcolor=bg]{python}
print("Resolveremos la ecuacion a*x**2 + b*x + c = 0")
a = float(input("Valor de a = "))
b = float(input("Valor de b = "))
c = float(input("Valor de c = "))
\end{minted}

Este peque\~no programa Python, al ser ejecutado con el comando \texttt{python test.py}, pregunta al usuario por los valores de las variables $a$, $b$ y $c$, que son asignadas como valores decimales (float). Ahora modifique el programa para que adem'as \textit{calcule e imprima} las dos soluciones de la ecuaci'on cuadr'atica, es decir, los valores 
\begin{equation}
x_\pm=\frac{-b\pm\sqrt{b^2-4ac}}{2a}.
\end{equation}
Para calcular la raiz cuadrada involucrada eleve el valor correspondiente la potencia $0.5$, es decir, use el hecho que $\sqrt{\alpha}=\alpha^{0.5}$.
\end{itemize}

\item Los caracteres individuales que forman una cadena alfanum'erica (string) pueden ser accesados usando el n'umero del \textit{'indice} correspondiente. En Python \textbf{el valor de los 'indices siempre comienza en 0}, luego 1, 2, etc. Para ilustrar esto, abra un int'erprete Python y ejecute los siguientes comandos:

\begin{minted}[bgcolor=bg]{python}
x = "Hola futuras legendas"
print(x[0])
print(x[1])
print(x[2])
print(x[3])
\end{minted}

\item Como comprob'o anteriormente la funci'on \texttt{len()} entrega el largo del string, es decir, el n'umero de caracteres que contiene. Por lo tanto

\begin{minted}[bgcolor=bg]{python}
print(x[len(x)-1])
\end{minted}

imprime el 'ultimo caracter del string, cuyo 'indice es \texttt{len(x)-1}, debido que el 'indice del primer caracter es 0. El mismo resultado, puede ser conseguido usando

\begin{minted}[bgcolor=bg]{python}
print(x[-1])
\end{minted}

Similarmente,

\begin{minted}[bgcolor=bg]{python}
print(x[-2])
\end{minted}

imprime el pen'ultimo caracter, y as'i sucesivamente. Por ejemplo, ejecute

\begin{minted}[bgcolor=bg]{python}
x[-3] == x[len(x)-3]
\end{minted}

para comprobar que se refieren al mismo caracter.

\item Tambi'en es posible acceder a un subconjunto de caracteres del string usando, en nuestro caso, \texttt{x[inicio:fin:paso]}, donde \texttt{inicio} y \texttt{fin} son los 'indices de los caracteres iniciales y finales y \texttt{paso} es un entero que define el paso. Si \texttt{paso} no es ingresado, el int'erprete considera que \texttt{paso = 1} . Por ejemplo, ejecute y verifique qu'e hacen los siguientes comandos

\begin{minted}[bgcolor=bg]{python}
print(x[0:4:1])
print(x[0:4])
print(x[5:16])
print(x[1:20:2])
\end{minted}

Note que el caracter correspondiente al 'indice \texttt{fin} \underline{NO} es desplegado. En lenguaje matem'atico podr'iamos decir que \texttt{x[inicio:fin]} suministra los caracteres de \texttt{x} con 'indices en el intervalo desde \texttt{inicio} \textit{cerrado} hasta \texttt{fin} \textit{abierto}.

\item Adem'as, si no se especifica \texttt{inicio} se asume el valor 0 (inicio del string) y si no se especifica \texttt{fin} se asume el valor \texttt{len(x)} (fin del string). Verifique esto ejecutando:

\begin{minted}[bgcolor=bg]{python}
print(x[:4])
print(x[5:])
print(x[5:-3])
print(x[::-1])
\end{minted}

\item ?`Qu'e hace cada uno de los siguientes comandos?, ?`Modifican el valor de \texttt{x}?

\begin{minted}[bgcolor=bg]{python}
x.upper()
x.replace("a","e")
x.find("f")
x.find("legend")
\end{minted}

\item Otro concepto muy importante en Python es el de \textit{listas}. Las listas son  similares a las cadenas, excepto que cada elemento puede ser de un tipo diferente. La sintaxis para crear listas en Python es [..., ..., ...]. Por ejemplo, ejecute:

\begin{minted}[bgcolor=bg]{python}
lista = [1, "hola", 1.0, 1-1j, True]
type(lista)
print(lista)
\end{minted}

Como puede ver, la variable \texttt{lista} es un nuevo tipo de objeto: `list'. En este caso, es una lista cuyos elementos son un entero, un string, un float, un complejo, y un booleano. Para verificar esto, imprima el valor y el tipo de cada elemento de la lista. Por ejemplo,

\begin{minted}[bgcolor=bg]{python}
print(lista[0],type(lista[0]))
print(lista[1],type(lista[1]))
\end{minted}

Este ejemplo tambi'en muestra que los 'indices de cada elemento de la lista son numerados de la misma manera que en un string:

\begin{minted}[bgcolor=bg]{python}
print(lista[0:3])
print(lista[::2])
\end{minted}

\item Los elementos de una lista pueden tener cualquier tipo reconocido por Python, por ejemplo, pueden ser otra lista!:

\begin{minted}[bgcolor=bg]{python}
superlista = ["cool",lista]
print(superlista)
\end{minted}

Imprima el valor y el tipo de cada elementos de esta lista. ?`Cu'antos elementos tiene la lista \texttt{superlista}? (respuesta, use la funci'on \texttt{len()}).

\item Existen diversas funciones en Python que crean listas. La funci'on \texttt{list()} crea una lista, por ejemplo, a partir de un string. Usando el string \texttt{x} definido anteriormente, ejecute

\begin{minted}[bgcolor=bg]{python}
y = list(x)
print(y)
print(type(y))
\end{minted}

\item Otra funci'on que crea listas 'utiles, esta vez de n'umeros \textit{enteros}, es \texttt{range(inicio,fin,paso)}, que crea una lista de valores desde \texttt{inicio} (cerrado) hasta \texttt{fin} (abierto!!), con paso \texttt{paso}. Ejecute,

\begin{minted}[bgcolor=bg]{python}
z = list(range(2,26,3))
print(z)
\end{minted}



\item ?`Qu'e hacen los siguientes comandos?, ?`Modifican el valor de \texttt{x} y/o \texttt{lista}?

\begin{minted}[bgcolor=bg]{python}
x.split(" ")
x.split("e")
lista.append("chao")
lista.insert(2,"cool")
\end{minted}
\end{questions}


\end{document} 